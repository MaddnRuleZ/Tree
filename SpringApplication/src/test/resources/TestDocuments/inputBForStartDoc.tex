% This is samplepaper.tex, a sample chapter demonstrating the
% LLNCS macro package for Springer Computer Science proceedings;
% Version 2.20 of 2017/10/04
%
\documentclass[runningheads]{llncs}
%
\section
    X420_1
    X
    X
    \subsection
        X420_2
        X
        X
        \begin{figure}
            X420_3
            X
            X
        \end{figure}
\section2
    X422
    X
    X

\begin{document}
%
\title{Active Learning for Regression with Drifting Oracles}
%
%\titlerunning{Abbreviated paper title}
% If the paper title is too long for the running head, you can set
% an abbreviated paper title here
%
\author{Anonym \orcidID{0000-1111-2222-3333} \and
Anonym \orcidID{1111-2222-3333-4444} \and
Anonym \orcidID{2222--3333-4444-5555}}
%
\authorrunning{A. Anonym et al.}
% First names are abbreviated in the running head.
% If there are more than two authors, 'et al.' is used.
%
\institute{Karlsruhe Institute of Technology, Karlsruhe, Germany}
%
\maketitle              % typeset the header of the contribution
%
\begin{abstract}
Current work on Active Learning (AL) tends to assume that the relationship between input and target variables does not change, i.e. the oracle is static.
But oracles can drift in many ways, making their formal description difficult.
Furthermore, input and target variables can be continuous, and measurements can be imprecise.
Current work on AL has not addressed these points in combination -- balancing data efficiency and estimation accuracy in such settings is challenging.
We formally describe drifting oracles for continuous variables and develop a new data synthesis model for common drift types.
We then propose a new AL model that can estimate its error due to drift.
Whenever the estimated error reaches a user-specified threshold, it performs measurements for self-recalibration.
We also propose a new kernel that allows our model to do this even with imprecise measurements.
Our drift-aware model requires up to 95\% fewer measurements than traditional drift correction methods.

\keywords{Concept Drift  \and Active Learning \and Integral Kernel \and Uncertain and Dynamic Environments.}
\end{abstract}