\part{Input}

\section{DAS HIER IST EIN KLEINER EINSCHUB}
    Das Schnabeltier (Ornithorhynchus anatinus) ist zweifellos eine erstaunliche und einzigartige Kreatur, die die Welt der Wissenschaft und die Neugier der Menschen seit seiner Entdeckung fasziniert hat. Mit seiner ungewöhnlichen Kombination von Merkmalen aus verschiedenen Tiergruppen ist es ein fesselndes Wesen, das viele Fragen aufwirft.

    \subsection{Physische Merkmale}
        Das Schnabeltier hat einen schlanken, torpedoförmigen Körper, der etwa die Größe einer Hauskatze erreicht. Sein dichtes und wasserabweisendes Fell ermöglicht es ihm, sowohl an Land als auch im Wasser zu leben. Auffällig ist sein schnabelartiger Schnauz, der dem eines Entenvogels ähnelt und mit Sinnesorganen ausgestattet ist, um elektrische Signale seiner Beutetiere wahrzunehmen. Die Vorderpfoten sind mit kräftigen Krallen versehen, die es beim Graben von Höhlen in Ufernähe nutzt. Die Hinterpfoten sind breit und mit Schwimmhäuten zwischen den Zehen versehen, was das Schnabeltier zu einem geschickten Schwimmer macht.

    \subsection{Lebensraum und Verhalten}
       Schnabeltiere sind hauptsächlich in Süßwasserhabitaten wie Flüssen, Bächen und Seen in Australien beheimatet. Sie bauen Burgen in den Ufernähe für Schutz und Unterschlupf. Das Schnabeltier ist hauptsächlich nachtaktiv und ernährt sich von kleinen aquatischen Invertebraten wie Insekten, Würmern und kleinen Krebstieren. Es nutzt seinen Schnabel, um Nahrung vom schlammigen Boden zu sammeln.

\section{Schutz und Bedrohungen}
    Trotz seiner faszinierenden Eigenschaften ist das Schnabeltier heute leider von verschiedenen Bedrohungen betroffen. Lebensraumverlust aufgrund von Wasserverschmutzung und -entnahme, sowie der Bau von Dämmen und Staudämmen beeinträchtigen die Lebensräume der Schnabeltiere. Darüber hinaus können sie in Fischernetzen gefangen und durch eingeführte Raubtiere wie Füchse und Hunde bedroht werden.

    \subsection{Schutzmaßnahmen}
    Um das Überleben des Schnabeltiers zu gewährleisten, sind Schutzmaßnahmen von entscheidender Bedeutung. Dazu gehören die Erhaltung und Wiederherstellung von Lebensräumen, die Kontrolle eingeführter Arten und die Sensibilisierung der Öffentlichkeit für die Bedeutung dieses einzigartigen Tieres für die Ökosysteme Australiens.
